\chapter{Benchmark}

The main aim of this thesis is to methodically examine the appearance phenomena that are frequently appearing in our day-to-day life but for some reason are still rarely implemented in the modern renderers. However, in this past few years, the need for the physically realistic renders has grown significantly and the implementations for these phenomena have been introduced. As they are still being consistently improved and integrated into modern mainstream renderers, it is absolutely necessary to have a testing suite which would properly evaluate their accuracy.

We propose a testing suite that contains a minimum number of test scenes which maximally exercice these implementations and an equivalent number of the reference images that we consider to be the ground truth, to our best knowledge. These are encapsulated in an automated workflow, which runs the tests with a single command and shows the results in form of a website. The suite also contains the data such as code snippets that can be easily integrated into any standard renderer.

\section{Framework}

\section{Supported Spectral Renderers}

\subsection{Mitsuba2}

\subsection{ART}

\section{Scenes}

\subsection{GGX Reflectance}

\subsection{Spectral accuracy}

\subsection{Polarization}

\subsection{Fluorescence}

\subsection{Iridescence}

\section{Common Data}

\section{Usage}

\section{Open source contributions}

\subsection{GGX for ART}

\subsection{Iridescence for Mitsuba2}

\subsection{Multi-channel EXR support for jeri.io}

\section{Future Work}